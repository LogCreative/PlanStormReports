\documentclass[10pt,a4paper,heading=false]{ctexart}
\ctexset{
    section = {
        number = \chinese{section},
        format = \bfseries\large\raggedright,
    }
}
\usepackage{gbt7714}
\begin{document}
    \title{\vspace{-3cm}“做题家”背后的教育差异问题
    \\\hspace*{15ex}\emph{\Large---基于城镇与城市的学生对比}
    }
    \author{\normalsize 班级: \underline{F1903301}\quad 姓名: \underline{李子龙}\quad 学号: \underline{518070910095}\quad 得分: \underline{\quad\quad\quad}}
    \date{}
    \maketitle
    \begin{abstract}
        “小镇做题家”是一群因单一评价体系,来到一流大学后屡屡受挫的青年学子。他们因中学时期受到较少的挫败感,在面临大学更为优秀的群体时产生了较大落差,有着“半城市化”初始阶段的心理。本文通过对城镇与城市中学阶段的学业负担上差异的分析,得出了“小镇做题家”的努力实际上并没有达到顶尖水平,是在与周围人的比较中得到了相对满足,在单一的赛道上沾沾自喜导致后来的严重挫败。目前国家已经实行“双减”、“强基计划”等政策,从根本上减少城乡教育差距,提升“小镇做题家”的发展潜能。

        \textbf{关键字}\quad 小镇做题家~~学业负担~~教育差异
    \end{abstract}
    \section{引言}
    % 重点解释标题
    % 小镇做题家身份建构困境 -- 个人的教育焦虑
    % 家长培养城市学生 -- 外界的教育焦虑
    “小镇做题家”一词源于豆瓣小组“985废物引进计划”的网友发帖,出身小城、埋头苦读但缺乏一定视野和资源的青年学子来到大城市一流高校,光环瓦解,屡屡受挫。\cite{solver}他们在低年级投入更多的时间用于学习时,身边的城市青年已经更积极于学生工作、社团活动、创新创业等领域,理想信念上也有差别。这种差距会造成社会交往上的障碍,“小镇做题家”又常常受到父母过高的期待而心理负担更重,一旦无法达到成绩预期,可能会陷入孤立无援的境地。\cite{solveranalysis}

    与城市青年的父母“鸡娃”教育焦虑不同,由于“小镇做题家”的父辈大多没有离开城镇眼界有限,这个符号背后更多的是个人教育焦虑。这种焦虑性质的差异所折射的教育水平差异值得探讨。
    \section{现状}
    % 挫败感少
    % 家庭相对优越
    近期对中国教育的研究表明,父或母若有大学学历,会对子女接受高等教育的概率提高40\%左右\cite{familyedu},也就指示着“小镇做题家”实际上一般出身于小镇物质条件较好的家庭,他们的父母有一定的见识、提供教育资本,希望子女能够在城市扎根立足。这种期望引导着这些进入一流高校的他们在中学时期于本地名列前茅,沉浸于“别人家的孩子”满足感之中,很少有挫败感。

    然而,对于一顶尖大学的生源分析指出,按家庭年收入推算,“小镇做题家”是在62.2\%统招生中63.08\%的一群人,也即40\%,相对于其他或城市或自主招生的考生而言属于少数群体,由于自主招生学生在统计意义上会比统招生在大一时的课程成绩高出1.4分\cite{pkurev},他们在大学成绩上更有可能被挫败。
    % 小镇做题家是一种“半城市化”的体现。
    % “飞地”较少,容易被孤立。
    % 父辈的融入与接替交给了这一代
    % 城乡二元结构和生活环境差异使农村大学生遭遇不平等的社会流动竞争

    目前,中国农民工之“半城市化”状况令人忧虑,他们虽然进入城市,找到了工作,但是没有融入社会,生活没有被有效支持,心理上会产生一种疏远乃至不认同的感受。\cite{paracivil}由于小镇父母将阶级晋升的希望交给了下一代,笔者认为,这些“小镇做题家”应当是新一代“半城市化”状况的初级阶段,尚未走向社会的他们因城乡二元结构和生活环境差异遭到不平等的社会流动竞争,原生家庭已无法再给予任何模板式的指引,当未来不确定时,就会迷茫。
    \section{分析}
    % 上海学生高一课后作业时间 3 小时以上的占50\%(问卷形式)
    % 小镇占33\%左右(真实情形)
    % 上海市 70\% 的课外辅导时间
    % 小镇仅依靠学校资源
    % - 超前学习(大学的、竞赛的)、功利化、短视化
    % - 题海战术,相信教育科学,提高学习效率
    % 学习类型上:自主招生学生、综合素质与视野(以及城市学生较多)
    % 单一的教育资源
    % 从结果来看,小镇的学习时间并没有到达转折点,
    % “小镇做题家”是单一的学习方向上努力,仍然带来相对好结果
    % 纵使在本地区的竞争中是佼佼者(期末考试的比较)
    % 自高考成绩上看也就是井底之蛙
    % 而多元的竞争让城市的孩子,在资本的裹挟下
    % 要承受更大的压力,并带了更长更先进的学习时间
    要分析产生“小镇做题家”现象的根本原因,就需要回到他们各自的中学阶段,比较学业负担上的差异。

    从2015到2017年,笔者在自己的城镇高中班级开展了作业监督计划,10周的统计的结果显示,全班每日作业时间超过3小时的同学占全班的33\%\cite{planstormoverview}。而同期上海市的类似调查结果显示上海市高一课后作业时间3小时以上的约占50\%\cite{shanghaihome}。由于两种统计方法不尽相同,笔者的调查方法是按照真实的作业完成量进行每周计算,而后者是采用网络问卷调查的方式全量统计,数据会有主观偏差。这样的话,城镇同学的作业时间与城市同学的作业基本一致或者略少一些。这个结论也在一定程度上符合了直觉,因为每个人的精力是有限的,城镇同学用于学习的时间与城市同学相比应该是大致相同的。
    
    这也就意味着,“小镇做题家”从统计意义上来看,实际上不是作业做得超多,而是所学习的知识单一,大部分都是课内知识。课外服务上,笔者所在的城镇中学晚上晚自习是会上课的,平均而言,会上近两个小时,依靠校内单一的校内教育服务,90\%以上是为了提升课内成绩。而与此形成对比的是,上海市的同期高一学生参加课外辅导的比例占67.28\%,虽然这其中75\%是为了查漏补缺,但是还是会有近25\%是为了超前学习与发展兴趣\cite{shanghaihome}。从这个意义上讲,课外的能力是有差距的,进入大学后这种差距也会被放大。

    笔者后来的分析表明,学生作业时间与近期的成绩提升没有显著效应($r=-0.024$)\cite{planstormrelation},这一点与同省内的另一所城镇高中得到的结论吻合($r=0.017$)\cite{phyquantity}。虽然上海市官方的调查结果也指示着学生学业成绩与教师布置作业时间的相关曲线基本呈倒“U”形,作业时间太少与作业时间太多,对学业成绩都有不利影响\cite{shanghaihome}。但是两者的结论略有区别,以及笔者进一步的分析显示,城镇学生作业时间却与高考成绩存在正效应关系\cite{planstormrelation},即城镇中学的学生作业量并没有大到以至于超过抑制长期成绩的转折点,佼佼者只能看到校内的成绩排名,实际如井底之蛙,并没有和同期城市学生一样会遇到实力非常强劲的对手,而拥有过度努力导致的长期成绩影响。

    这些印证着,“小镇做题家”很大程度上相对周围人是努力的,但是从全国的水平比较来看,并没有努力到顶尖的水平,更像是在单一的赛道上沾沾自喜。这也就是“小镇做题家”在入学前很少有挫败感,而到进入一流大学后会有挫败感更加根本的原因。
    \section{对策}
    % 双减,解除资本的裹挟、回归教育本质
    % 本人的尝试:灵活的作业规划,过渡的尝试

    对于大学而言,对于“小镇做题家”在直接上应当加强引导,加强物质关怀与思想关怀,并在一定程度上给予一些先修的支持,以更好地适应大学生活。
    
    对于中学而言,从根本上应当引入更加先进的教学理念与教学方法,科学利用课余时间,避免完全功利化、短视化的学习让更多的“小镇做题家”丧失对知识本身的乐趣与探索欲望。笔者在当年发现这个问题之后,也在当地进行了一些过渡上的实践\cite{planstormtech},对城镇同学由负向惩罚到正向鼓励的过渡,提升更为本质的学习动力,逐渐摆脱对老师规划作业的依赖,形成自己的一套方法,并适当发展课外的学习,提升自己的知识面,将其纳入作业时间的范畴。这个方法相比于粗犷式的“填鸭式”教育------老师直接给你安排学习时间这一方向是有非常大区别的,显然后者的对同学未来的不利影响更大。

    对于城市的同学而言,最近国家出台了“双减”政策\cite{doublereduction},政策明确要求坚持从严治理,全面规范校外培训行为。政策对教育资本化是一个很好的遏制,很多资本化教育机构应声倒闭,这有助于城市学生减少课外辅导的高比例,让教育回归本质,满足学生个性化、差别化、实践性学习的需求,使得教育不再成为奢侈品,成为少数有钱人的专利\cite{revolution}。

    国家在2020年之后,终止了“自主招生”政策,转而使用“强基计划”作为高校新的创新人才引入落脚点。虽然“自主招生”招进了一些优秀的“偏才”、“怪才”,但是在其他条件固定的前提下,第一代大学生在自主招生选拔机制中成功的概率,比第二代大学生减少39.3\%,也就是其依然存在向知识精英阶层、城市学生倾斜的精英化趋势\cite{fair}。而“强基计划”中明确高考成绩占比不低于85\%,更加注重高考人才选拔的公平公正性,更好地向国家未来在基础学科输送顶尖创新人才\cite{newrev}。

    这些归根到底是缩小城乡的教育差距、也是缩小城乡差距的举措,在代代的改善中,通过受教育逐步实现阶级跃迁。“半城市化”的可行性路径暗示着,这些“小镇做题家”可以考虑转而向中小城市“融入”和“接替”,而没有必要去接纳无法承受的大城市之苦\cite{paracivil}。当然,国家层面也应当加强中小城市的配套基础设施建设,使其不断满足新市民增长的需要,从而一步一步地减少城乡差距,逐步实现城市化。

    \section{结论}
    “小镇做题家”是一个因中学与城市的学业水平差距,大学时没有足够的支持而产生的现象。这一现象背后的教育水平差距问题已经是中国教育问题的折射之一,解决之可以更好地促进阶级流动性,促进社会的进步与发展。

    严格落实国家出台的“双减”、“强基计划”政策将有助于减少城乡的教育差异,从大学到中学到当地部门的各个机构也应当从内部不断地试验改革,发现“小镇做题家”的更多发展潜能,并为其未来的发展提供必要的支持。

    \bibliography{ref}
\end{document}